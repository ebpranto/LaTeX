\documentclass[11pt, a4paper]{article}
 % the space between documant class and begin document is call preamble
% \usepackage{fullpage} % for set 1 inch page mergin

% \usepackage[top = 1in, bottom = 1in, left = 0.5in, right = 0.5in, paperwidth=8.5in, paperheight=11in]{geometry} % for set custom mergin

\usepackage[margin = 1in]{geometry}
\usepackage{amsfonts, amssymb, amsmath}
\usepackage{tikz}
\usepackage{pgfplots}
\usepackage{fontawesome5} %calculator, pen, note etc emoji




%Create macros:
\def\eq1{\dfrac{x}{3x^2+x+1}}

\newcommand{\set}[1]{\setlength{\itemsep}{#1cm}}



\begin{document}

$\frac{x}{3x^2+x+1}$

$\dfrac{x}{3x^2+x+1}$ % larger fraction

\section{Introduction}
% \textbf{Introduction}
\begin{enumerate}
    \set{0.3}
    \item \faCalculator\ Let's examine the function: $\eq1$  %use macros
    \item \faPen This is symbol of set of all real numbers: $\mathbb{R}$.
    \item \faEdit This is symbol of set of all Integers: $\mathbb{Z}$.
    \item This is symbol of set of all rational numbers: $\mathbb{Q}$.
    \item  What’s	better	than	one	model	making	predictions?	Well,	how	about	a	bunch	of
 them?	Ensembling	is	a	technique	that	is	fairly	common	in	more	traditional
    \item The
 idea	is	to	obtain	a	prediction	from	a	series	of	models,	and	combine	those
 predictions	to	produce	a	final	answer
    \item  There	are	plenty	of	approaches	to	ensembles,	and	we	won’t	go	into	all	of	them
 here.	Instead,	here’s	a	simple	way	of	getting	started	with	ensembles,	one	that	has
 eeked	out	another predictions
    
\end{enumerate}

\end{document}