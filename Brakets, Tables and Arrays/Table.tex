\documentclass[11pt]{article}

\usepackage{float} %install package


\parindent 0px % set indentent
\pagestyle{empty} % remove page numbers

\begin{document}
Tables:\\
% c for center, l- left, r-right
\begin{tabular}{|c||c|c|c|c|c|} \hline
$x$ & 1 & 2 & 3 & 4 & 5\\ \hline
$f(x)$&10&11&12&13&14\\ \hline
\end{tabular}

\vspace{1cm}

%Table:1
\begin{table}[H]

%center the table
\centering

\def\arraystretch{1.5} % extra padding: space from element to top and bottom 
\begin{tabular}{|c||c|c|c|c|c|} \hline
$x$ & 1 & 2 & 3 & 4 & 5\\ \hline
$f(x)$&$\frac{1}{2}$&11&12&13&14\\ \hline
\end{tabular}

% set caption
\caption{These values represents the functions}

\end{table}


%Table:2
\begin{table}[H]

%center the table
\centering
% set caption
\caption{The relationship between $f$ and $f'$} 
\def\arraystretch{1.5} % extra padding: space from element to top and bottom 
\begin{tabular}{|l|p{3in}|} \hline % p - paragraph
$f(x)$ & $f'(x)$\\ \hline
$x>0$& The function $f(x)$ is increasing.The function $f(x)$ is increasing.The function $f(x)$ is increasing.The function $f(x)$ is increasing.\\ \hline
\end{tabular}



\end{table}

\end{document}